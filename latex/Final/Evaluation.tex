\section{Evaluation}

We evaluated the classifiers from each branch separately (and the recommender system in the profile-based branch).

\subsection{Review-Based Branch}
Due to the large size of the entire dataset (over 1.5 million reviews), we systematically divided the dataset in our evaluation. Moreover, because a part-of-speech tagger significantly increases the running time of extracting all of our features, we measured our classifiers' performances when they contained and did not contain the adjective and adverb count features separately. We first randomly chose 100,000 reviews from the full dataset, from which we randomly selected 70,000 to be the training set and 30,000 to be the test set. The features not including the adjective and adverb counts were then extracted from the training set, and then the classifiers were trained on these features and evaluated on the test set. The process of randomly choosing and splitting 100,000 reviews and training and evaluating the classifiers was repeated 9 more times, and the results were averaged over the 10 trials.

We then randomly chose 10,000 reviews from the full dataset, from which we randomly selected 7,000 to be the training set and 3,000 to be the test set. The full set of features was then extracted from this smaller training set, and the classifiers were again trained and evaluated. As in the larger dataset, we performed a total of 10 trials. The smaller dataset allowed for the features, specifically the adjective and adverb count, to be extracted in a reasonable amount of time. Separating the dataset like this also allowed us to examine the impact the adjective and adverb count had on classification.

To evaluate the performance of the classifiers, we recorded the general accuracy of each classifier (the percentage of correctly-labeled reviews), the accuracy of each classifier when predicting whether or not a review is five stars, the accuracy of each classifier when predicting whether a review is one star, five stars, or neither, and the ``soft'' accuracy of each classifier, in which the classifier is said to have made a correct prediction if the predicted label is at most one star off from the actual label. These types of accuracy allow us to look at different aspects of the classifier -- how well did it do in general, how well did it perform on predicting polarized reviews, and how often was it close enough to the actual label? This accounts for the possibility that the features were better suited for detecting polarized reviews (one star or five stars) rather than choosing one of the five class labels, as well as the possibility that many reviewers left reviews that were inconsistent with their final star rating from another perspective. These are important to consider, since it has been found that people online tend to report extreme feelings rather than average feelings, and positive feelings rather than negative feelings \cite{imbalance}.

\subsection{Profile-Based Branch}

\subsubsection{Classifier Evaluation}

\subsubsection{Recommender System Evaluation}